% SPDX-FileCopyrightText: 2026 IObundle, Lda
%
% SPDX-License-Identifier: MIT
%
% Py2HWSW Version 0.81.0 has generated this code (https://github.com/IObundle/py2hwsw).

LRU & M & ? & 0 & ? & Least Recently Used -- more resources intensive - N*log2(N) bits per cache line - Uses counters \\* \nobreakhline
\rowcolor{iob-blue}
PLRU\_MRU & M & ? & 1 & ? & bit-based Pseudo-Least-Recently-Used, a simpler replacement policy than LRU, using a much lower complexity (lower resources) - N bits per cache line \\* \nobreakhline
PLRU\_TREE & M & ? & 2 & ? & tree-based Pseudo-Least-Recently-Used, uses a tree that updates after any way received an hit, and points towards the oposing one. Uses less resources than bit-pseudo-lru - N-1 bits per cache line \\* \nobreakhline
\rowcolor{iob-blue}
WRITE\_THROUGH & M & ? & 0 & ? & write-through not allocate: implements a write-through buffer \\ \hline
WRITE\_BACK & M & ? & 1 & ? & write-back allocate: implementes a dirty-memory \\ \hline
\rowcolor{iob-blue}
ADDR\_W\_CSRS & M & ? & 5 & ? & Address width of CSRs \\ \hline
FE\_ADDR\_W & P & 1 & 24 & 64 & Front-end address width (log2): defines the total memory space accessible via the cache, which must be a power of two. \\ \hline
\rowcolor{iob-blue}
FE\_DATA\_W & P & 32 & 32 & 64 & Front-end data width (log2): this parameter allows supporting processing elements with various data widths. \\ \hline
BE\_ADDR\_W & P & 1 & 24 &  & Back-end address width (log2): the value of this parameter must be equal or greater than FE\_ADDR\_W to match the width of the back-end interface, but the address space is still dictated by ADDR\_W. \\ \hline
\rowcolor{iob-blue}
BE\_DATA\_W & P & 32 & 32 & 256 & Back-end data width (log2): the value of this parameter must be an integer  multiple \$k\textgreater =1\$ of DATA\_W. If \$k\textgreater 1\$, the memory controller can operate at a frequency higher than the cache's frequency. Typically, the memory controller has an asynchronous FIFO interface, so that it can sequentially process multiple commands received in paralell from the cache's back-end interface.  \\ \hline
NWAYS\_W & P & 0 & 1 & 8 & Number of cache ways (log2): the miminum is 0 for a directly mapped cache; the default is 1 for a two-way cache; the maximum is limited by the desired maximum operating frequency, which degrades with the number of ways.  \\ \hline
\rowcolor{iob-blue}
NLINES\_W & P &  & 7 &  & Line offset width (log2): the value of this parameter equals the number of cache lines, given by 2**NLINES\_W. \\ \hline
WORD\_OFFSET\_W & P & 1 & 3 &  & Word offset width (log2):  the value of this parameter equals the number of words per line, which is 2**OFFSET\_W.  \\ \hline
\rowcolor{iob-blue}
WTBUF\_DEPTH\_W & P &  & 4 &  & Write-through buffer depth (log2). A shallow buffer will fill up more frequently and cause write stalls; however, on a Read After Write (RAW) event, a shallow buffer will empty faster, decreasing the duration of the read stall. A deep buffer is unlkely to get full and cause write stalls; on the other hand, on a RAW event, it will take a long time to empty and cause long read stalls. \\ \hline
REP\_POLICY & P & 0 & 0 & 3 & Line replacement policy: set to 0 for Least Recently Used (LRU); set to 1 for Pseudo LRU based on Most Recently Used (PLRU\_MRU); set to 2 for tree-based Pseudo LRU (PLRU\_TREE). \\ \hline
\rowcolor{iob-blue}
WRITE\_POL & P & 0 & 0  & 1 & Write policy: set to 0 for write-through or set to 1 for write-back. \\ \hline
USE\_CTRL & P & 0 & 0 & 1 & Instantiates a cache controller (1) or not (0). The cache controller provides memory-mapped software accessible registers to invalidate the cache data contents, and monitor the write through buffer status using the front-end interface. To access the cache controller, the MSB of the address mut be set to 1. For more information refer to the example software functions provided. \\ \hline
\rowcolor{iob-blue}
USE\_CTRL\_CNT & P & 0 & 0 & 1 & Instantiates hit/miss counters for reads, writes or both (1), or not (0). This parameter is meaningful if the cache controller is present (USE\_CTRL: 1), providing additional software accessible functions for these functions. \\ \hline
AXI & M & NA & NA & NA & AXI interface used by backend \\ \hline
\rowcolor{iob-blue}
AXI\_ID\_W & P & 0 & 1 & 32 & AXI ID width \\ \hline
AXI\_ID & P & 0 & 0 & 32 & AXI ID \\ \hline
\rowcolor{iob-blue}
AXI\_LEN\_W & P & 0 & 4 & 32 & AXI length \\ \hline
AXI\_ADDR\_W & P & 0 & BE\_ADDR\_W & 32 & AXI address width \\ \hline
\rowcolor{iob-blue}
AXI\_DATA\_W & P & 0 & BE\_DATA\_W & 32 & AXI data width \\ \hline
VERSION & M & NA & 24'h000701 & NA & Product version in SemVer format. This 24-bit macro uses nibbles to represent decimal numbers using their binary values. The two most significant nibbles represent the major part of the version, followed by two nibbles that represent the minor part. The two least significant nibbles represent the patch version. For example V12.34.56 is represented by 0x123456. \\
